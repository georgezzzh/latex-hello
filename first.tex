\section{安装教程}
详细参考这个网站, www.latex-tutorial.com
\subsection{TexWorks字体设置}
TexWorks中编辑-首选项-编辑器默认配置(字体),设置好之后,重启生效。
\section{一级标题}
section, table of content 是目录的意思。
\subsection{二级标题}
subsection.
\subsubsection{三级标题}
subsubsection. 
\paragraph{段落}
paragraph
\subparagraph{子段落}
Subparagraph, 
LaTex中层次分为5层,[section, subsection, subsubsection, paragraph, subparagraph],五个层次,可以在设置目录中显示要在目录中显示的层次,其中0代表什么都不显示,5代表显示到subparagraph
% 单独的设置目录的深度为1
\addtocontents{toc}{\setcounter{tocdepth}{1}}
\section{第二个标题,Another section}
\subsection{第二个子标题 Eat}
Eatting is necessary for human!
% 重新设置目录的深度为3
\addtocontents{toc}{\setcounter{tocdepth}{3}}
\section{数学公式, Math}
This is math envirment.
\begin{equation}
f(x)=x^2
\end{equation}
The following equation use amsmath package feature, the feature is that there isn't number beside equation.
\begin{equation*}
g(x)=\sum_{i=1}^{100}i
\end{equation*}
\section{数学公式}
接下来写几个数学公式,描述数学公式在LaTex中的使用
\subsection{行内公式写法,用\$作定界符号}
嵌入文本中的公式用\$包围,e.g. $1=\frac{1}{1}$ 嵌入完毕
\subsection{公式环境}
有equation和align两种环境, equation环境用于一个公式的排版,align可以写多行公式,会根据\&符号的位置对齐上下两个式子,$\backslash\backslash$用来换行。

\begin{equation}
1+2=3
\end{equation}

\begin{align}
1+0 &=1\\
1&=2-1
\end{align}
以下列举几个公式的例子\\
\subsubsection{积分}
\begin{align}
F(x)=\int_a^b\frac{1}{\sqrt{3}}x^3
\end{align}
\subsubsection{矩阵}
矩阵,用\$号界定的环境之下,用\{matrix\}环境写入
$
\begin{matrix}                                      
1 & 0\\
0 & 1
\end{matrix}
$

当矩阵要写入左右括号时,引入$\backslash left[$会放大括号
$\left[
\begin{matrix}
1 & 0\\
0 & 1
\end{matrix}
\right]
$
\subsubsection{$\backslash$left(的作用}
输入普通的大括号(普通字符)
\begin{equation}
(\frac{1}{\sqrt{x}}) 
1=2+-1
\end{equation}

$\backslash$left(
\begin{equation}
\left( \frac{1}{\sqrt{x}} \right)
\end{equation}
% 插图章节
\section{插图}
插一张巫师3的配图
% 为图片环境设置浮动值,此设置能使得图片在当前tex位置
% h是(here)-与tex文档相同位置, t(top of page), b(bottom of page), p (on an extra page), !(will force the specified location,强制执行制定的命令)
\begin{figure}[h!]
	\includegraphics[width=\linewidth]{avatar.png}
	\caption{picture about my avatar}
	%主要是为了标注一个图片,方便于引用
	\label{fig:qqavatar}
\end{figure}  

\subsection{引用图片示例 }
Figure \ref{fig:qqavatar} shows a picture on here!
\subsection{subfigure子图}
使用子图,需要用usepackage\{subcaption\}包, 此外就是subfigure环境了
\begin{figure}[h!]
	\centering
	\begin{subfigure}{0.4\linewidth}
		\includegraphics[width=\linewidth]{avatar.png}
		\caption{avatar}
	\end{subfigure}
	\begin{subfigure}{0.4\linewidth}
		\includegraphics[width=\linewidth]{avatar.png}
		\caption{also avatar}
	\end{subfigure}
	\begin{subfigure}{0.6\linewidth}
		\includegraphics[width=\linewidth]{avatar.png}
		\caption{avatar copy}
	\end{subfigure}
	\caption{two character of qq avatar}
	\label{fig:avatar}
\end{figure}


% 引用章节
\section{引用BibTeX}
随便的引用的BOOK \cite{DUMMY:1} ,嵌入在文本中;引用的ARTICLE \cite{ARTICLE:1},引用的INBOOK例子\cite{BOOK:2},引用因特网WEBSITE\cite{WEBSITE:2}的\cite{WEBSITE:1}
% 脚注章节
\section{脚注}
This is some example text\footnote{\label{myfootnote}脚注的具体内容,写在页面最下方}。textsuperscript是在文本右上有一个小角标,我在这里再次引用上面提到的脚注\textsuperscript{\ref{myfootnote}}。
% table章节
\section{table}
LaTeX中表格通过table环境和tabular环境的结合来创建。table环境负责定义表格的位置和对齐方式。表格真正的内容包括在tabular环境中。textbf\{\}是用来划定列数的。用在首行。
\begin{table}[h!]
	\begin{center}
		\caption{Your first table} %表格的说明文字
		\label{tab:table1}
		\begin{tabular}{l|c|r|l}
			\textbf{value 1} & \textbf{value 2} & \textbf{value 3} & \textbf{value 4}\\
			\hline
			1 & 1110.1 & a & 100 \\
			\hline
			2 & 10.1    & b  & 7
		\end{tabular}
	\end{center}
\end{table}
\subsection{跨多行的表格}
跨多行的表格,需要用usepackage\{multirow\},  multirow\{NUMBER OF ROW\} \{宽度( $\ast$为自动计算)\}\{内容\}
\begin{table}[h!]
	\begin{center}
		\caption{support table of multirow cell } %表格的说明文字
		\label{tab:table2}
		\begin{tabular}{l|c|r|l} %l表示left对齐, c表示center, r表示right
			\textbf{value 1} & \textbf{value 2} & \textbf{value 3} & \textbf{value 4}\\
			\hline
			\multirow{2}{*}{12} &1111.2 & d & 10\\
			                                     & 1110.1 & a & 100 \\
			\hline
			2 & 10.1    & b  & 7
		\end{tabular}
	\end{center}
\end{table}
\subsection{跨多列的表格}
多列控制命令, multicolumn\{Number of column\}\{Alignment,对齐方式\}\{content\}
\begin{table}[h!]
	\begin{center}
		\caption{Your first table} %表格的说明文字
		\label{tab:table3}
		\begin{tabular}{l|c|r|l}
			\textbf{value 1} & \textbf{value 2} & \textbf{value 3} & \textbf{value 4}\\
			\hline
		      \multicolumn{2}{c|}{120}  & a & 100 \\
			\hline
			2 & 10.1    & b  & 7
		\end{tabular}
	\end{center}
\end{table}
\subsection{多行多列结合的表格}
具体使用看以下code
\begin{table}[h!]
	\begin{center}
		\caption{Your first table} %表格的说明文字
		\label{tab:table4}
		\begin{tabular}{l|c|r|l}
			\toprule
			\textbf{value 1} & \textbf{value 2} & \textbf{value 3} & \textbf{value 4}\\
			\hline
			\multicolumn{2}{c|}{\multirow{2}{*}{1234}} & a & 100 \\
			\multicolumn{2}{c|}{} &c & 101\\
			\hline
			2 & 10.1   & b  & 7\\
			\bottomrule %该命令画出的线是粗的,而hline是细的,注意在最后画线的时候,上一行要进行换行\\
		\end{tabular}
	\end{center}
\end{table}

\subsection{旋转表格}  %不太懂为什么旋转表格总是新起一页打开
使用sidewaystable环境可以解决表格横向打印。
\begin{sidewaystable}[ph!]
	\begin{center}
		\caption{Landscape table.} %表格的说明文字
		\label{tab:table5}
		\begin{tabular}{l|c|r}
			\toprule
			\textbf{value 1} & \textbf{value 2} & \textbf{value 3}\\
			\midrule
			1 & 1110.1 & a  \\
			\hline
			2 & 10.1    & b  \\
			\bottomrule
		\end{tabular}
	\end{center}
\end{sidewaystable}
\subsection{超长表格}
有些表格是跨页的,使用usepackage\{longtable\}可以解决这个问题。具体参考网页\footnote{\label{multipage_table}https://www.latex-tutorial.com/tutorials/tables/}中的Multipage tables.